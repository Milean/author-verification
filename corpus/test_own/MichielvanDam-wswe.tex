\documentclass{llncs}
\usepackage{url}

% \section{This is a First-Order Title}
% \subsection{This is a Second-Order Title}
% \subsubsection{This is a Third-Order Title.}
% \paragraph{This is a Fourth-Order Title.}

\bibliographystyle{splncs}

\begin{document}

\title{Reflections on Trust on the Web}
\subtitle{Reaching an accurate trust-based recommender system}
\author{Michiel van Dam}
\institute{Web and Semantic Web Engineering, Delft Technical University
\email{m.c.vandam@student.tudelft.nl}}

\maketitle

\begin{abstract}
Trust is a concept from everyones lives, which influences many important choices, such as which articles to purchase and from whom. In this article the classical concept of trust is evaluated and applied to the world wide web. A broad reflection on how trust is used on the web is given, and special attention is paid to how personalized recommendations can be given using trust modeling. This last question is evaluated to reach several research questions that could be used as starting point for new research. Furthermore suggestions are done on how to answer the formulated questions. 
\end{abstract}

%-----------------------------------------------------------------------

\section{Introduction}
Trust is an everyday concept. People trust their families, their friends and their colleagues in what they say or advise. People also tend to trust books, news media, or other public available information. This trust in others, and the advice that is given from a trusted source, influences many economic decisions, such as which store to buy from, which stock options to take or which person to hire for a new job. This makes trust and determining what makes a source trustworthy important concepts to research.\\
\\
With the creation of the world wide web (WWW) a new source for information and advice was created. Especially after the emergence of Web 2.0 and user-generated content people had access to the opinions and advice of many others, which they would previously not have known. While previously people would have known these others because they shared an activity or a friend, the WWW provides no such reference, as information and user-generated reviews can be found after a search for the item or activity that is being considered. In the semantic web several parties can generate different or even contradicting information, making the question of what authors to trust even more important for finding the right information.

\subsection{Current use of trust on the web}
Currently trust is an important concept when browsing the web. On the technical side public key infrastructure is used to authenticate websites and to signal to users that the website they visit can be trusted. This all depends on a shared trust in certificate authorities that gets propagated to websites whose certificates they sign. This trust serves as a measure of authenticity rather than a measure of quality.\\
\\
Less technical and more information-oriented, trust also serves as a measure of quality for information.\cite{Artz2007} Trust in this sense can be indicated in multiple ways, for example by other sources referring to high-quality information sources by citing it or linking towards it. In a sense this is equal to a good reputation, as already used by algorithms like Google Pagerank.
When this information concerns reviews or comments posted by other users, websites can use rating systems to determine the best contributions. Two examples of this are the rating system of the IMDB, going from one star to ten stars, to compute an average ranking of contributions, or the ranking system for Amazon product reviews: "120 of 200 people found the following review helpful". In these examples, giving a high ranking to a movie or signaling you found a review helpful is an implicit expression of trust towards the source of the review.\\
\\
More high-level, trust is used by websites to compel users to share their personal details; most users will not reveal personal data to websites, unless they trust the supplier of that website and trust him to respect their privacy and keep their information safe.\cite{Hoebel2011}\\
\\
Finally trust is connected to website design. When dealing with an unknown vendor or website users have to determine whether they trust this vendor in the time they browse that vendors website before making their first purchase.\cite{HarrisonMcKnight2002} In this sense, trust can be generated by a good appearance and initial interactions.\\
\\
As the main contribution of this article, research questions will be formulated focused on the following question: How to model trust in a way that allows for accurate recommendations? To this purpose the remainder of this article proceeds as follows. Section 2 discusses how to model trust relations on the web and what decisions have to be made. Section 3 then discusses related work and questions that have already been answered. Section 4 formulates several research questions that have yet to be answered, and gives a short discussion on how to perform the needed research. Finally section 5 provides a short conclusion about trust on the web and the questions still left unanswered.

\section{Trust modeling}
Where the previous section discussed how trust is used, this section focuses more on the concept of trust, provididing a definition and several aspects of trust that need to be taken into account for an accurate model.     

\subsection{Aspects of trust}
The oxford english dictionary gives the following definition of trust: \emph{A firm belief in the reliability, truth, or ability of someone or something.}\cite{Press} Based on this definition four aspects of trust can already be identified.

\begin{itemize}
\item Trust requires \textbf{two parties}. Every trust relation needs to have a someone or something that is trusted, and someone or something to do the trusting. As such, trust is the relationship $t(a,b)$, meaning that $a$ trusts $b$ to some extent.
\item Trust is \textbf{asymmetrical}. The given definition doesn't specify a reciprocal relation. Person $a$ can trust person $b$ without $b$ also trusting $a$.
\item Trust needs a \textbf{context}. When believing in the ability of someone, it concerns a specific area of ability. $A$ can trust $b$ when it comes to economic decisions, but this does not imply that $a$ has the same trust when it comes to moral decisions.
\item Trust has a \textbf{firmness}. The firmess of the belief reflects to what extent a person or thing is trusted. This can be modeled using a weight which can be either discrete or continuous.
\end{itemize}

\subsection{Modeling decisions}
The aspects of trust already mentioned above give an outline for modeling trust, and based upon those aspects trust can be modeled as a directed weighted graph. However, it still leaves several questions unanswered, which need to be decided upon for any complete modeling of trust.

\begin{itemize}
\item To what extent is trust \textbf{timebound}? Trust in a certain person or thing hasn't always existed. Policies of companies can change and so can the trust in them. People change over the years and so will their preferences. This makes trust inherently timebound.

\item Is trust fully \textbf{transitive}? When $a$ trusts $b$, and $b$ trusts $c$, will $a$ consequently also trust $c$? How is a propagation of trust modeled to properly reflect reality?

\item How to model \textbf{distrust}? Is distrust taken into account in the firmness of the belief? In this sense, distrust can be modeled by a negative number or by a zero. Can it be handled in the same way as trust, or is it less transitive or timebound? Can person $a$ both trust and distrust person $b$, each in different amounts?
\end{itemize}

%-----------------------------------------------------------------------

\section{Related work}

Many important contributions have already been made to the field of trust modeling, some originating from the social sciences and some more from the viewpoint of recommender systems. It was already shown by \cite{Sinha2001} that people rely more on recommendations from trusted parties, than on recommendations done automatically by a recommender system, clearly indicating the need for this research in the domain of the WWW. An extensive survey is done by \cite{Artz2007}, defining four broad areas of how trust is used in computing and on the web: policy-based, reputation-based, creating general models, and trust in information resources. The third and fourth area are mostly considered for this article.\\
\\
In \cite{HarrisonMcKnight2002} the impact of initial interactions with a website on trust is evaluated, showing that website quality and reputation are important factors in determining initial trust. Furthermore in \cite{Hoebel2011} trust and privacy concerns of users are researched when considering to reveal personal information, concluding that most people are very hesitant to reveal personal information to untrusted websites.\\
\\
For the transitivity of trust and the modeling of distrust, \cite{Guha2004} provides several models of propagating trust and distrust and evaluates these models on a large test-set. This shows that trust is more important than distrust to propagate, but that adding distrust in the model significantly improves the accuracy. Finally \cite{Victor2011} discusses recommendations and suggests promising models for recommendations based on trust and distrust expressions. While these models provides a small increase in accuracy over purely content-based recommendations, the results are not significant enough to provide an important step forward.

%-----------------------------------------------------------------------

\section{Research questions}

To come to a succesful recommender system based on trust modeling, the question needs to be answered how to accurately recommend based on trust, and for what purposes. A clear example of a purpose for which a trust-based recommendation would be useful is when viewing reviews for an item or activity that is considered for purchasing. The question how to recommend based on trust can be further evaluated by answering the following research questions:

\begin{itemize}
\item \textbf{How to accurately recommend based on trust?}
	\begin{enumerate}
	\item What metrics can be used to determine the trust a party $a$ has in $b$?
	\item How can distrust best be modeled to increase the accuracy of recommender systems?
	\item How can context-specific trust be taken into account for general recommender systems?
	\end{enumerate}
\end{itemize}

Research question (1) is relevant because trust can be expressed in many ways. On twitter a person can 'follow' another person, but does this indicate trust, and if so, how much? When a review on Amazon is labeled as 'Not helpful', does that indicate indifference or distrust? Defining and evaluating metrics for trust seems an important step to come to a recommender system.

For researching question (1) a specific domain will have to be chosen. On each website users can express trust in different ways, so the scope of this research will be limited to the platforms under evaluation, possibly ending at some global conclusions that can be applied to other similar platforms. The methodology can use cross-verification by, for example, calculating the possibility that the user would get his current behavior recommended, without taking that specific current behaviour into account for the calculation.\\
\\
Research question (2) is focused on a topic that \cite{Guha2004} already partially answered, but remains nonetheless relevant. First off, the results from 2004 can be outdated as the use of the web can have evolved since then, and a more recent evaluation of the impact of distrust on recommendations, based on the same datasource, failed to provide conclusive results.\cite{Victor2011} Secondly and more importantly, the model was only provided for epinions.com and was not evaluated on other sources, possibly limiting the scope of the research.

For researching this question the same methodology used by \cite{Guha2004} can be taken to confirm their results on a more recent dataset. If available, it can also be applied to datasets originating from other sources, to conclude which of the findings are specific to epinions.com and which seem to hold in general.\\
\\
Finally research question (3) is not addressed by current research yet. Search engines such as Google provide personalized search results, for which a trust-based improvement can possibly be made. Here it becomes important to model in which context it holds that $a$ trusts $b$, and to differentiate results based on the context. In this regard, the semantic web has an important contribution, as it can more easily provide context for information. 

To research this, first a model needs to be developed for taking the context into account when determining trust. Depending on the envisioned model several tests can be done on user behavior in search engines, comparing their favoured results to the results suggested by revised recommender systems that take the new model into account.

%-----------------------------------------------------------------------

\section{Conclusions}
The conceptual notion of trust was evaluated and compared to how it is used on the web. Especially distrust and context were identified as unsolved issues in digital trust, and three research questions were formulated as a starting point for further research. While the modeling of distrust and deciding on platform-specific metrics for trust can be more easily answered, using context-specific trust in general recommender engines might prove more complex and can remain an interesting field of research for the next few years.

%-----------------------------------------------------------------------

\bibliography{references}{}

\end{document}