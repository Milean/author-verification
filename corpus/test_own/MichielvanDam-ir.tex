
%%%%%%%%%%%%%%%%%%%%%%% file typeinst.tex %%%%%%%%%%%%%%%%%%%%%%%%%
%
% This is the LaTeX source for the instructions to authors using
% the LaTeX document class 'llncs.cls' for contributions to
% the Lecture Notes in Computer Sciences series.
% http://www.springer.com/lncs       Springer Heidelberg 2006/05/04
%
% It may be used as a template for your own input - copy it
% to a new file with a new name and use it as the basis
% for your article.
%
% NB: the document class 'llncs' has its own and detailed documentation, see
% ftp://ftp.springer.de/data/pubftp/pub/tex/latex/llncs/latex2e/llncsdoc.pdf
%
%%%%%%%%%%%%%%%%%%%%%%%%%%%%%%%%%%%%%%%%%%%%%%%%%%%%%%%%%%%%%%%%%%%


\documentclass[runningheads,a4paper]{llncs}

\usepackage{amssymb}
\setcounter{tocdepth}{3}
\usepackage{graphicx}

\usepackage{url}
\urldef{\mails}\path|m.c.vandam@student.tudelft.nl|
\newcommand{\keywords}[1]{\par\addvspace\baselineskip
\noindent\keywordname\enspace\ignorespaces#1}

\begin{document}

\mainmatter  % start of an individual contribution

% first the title is needed
\title{Survey: A comparison of markov models and n-grams for spam filtering techniques}

% a short form should be given in case it is too long for the running head
\titlerunning{Survey: spam filtering techniques}

% the name(s) of the author(s) follow(s) next
%
% NB: Chinese authors should write their first names(s) in front of
% their surnames. This ensures that the names appear correctly in
% the running heads and the author index.
%
\author{Michiel van Dam}
%

% the affiliations are given next; don't give your e-mail address
% unless you accept that it will be published
\institute{Information Retrieval, Delft University of Technology,\\
Mekelweg 4, 2628 CD  Delft, Netherlands\\
\mails\\
\url{http://ewi.home.tudelft.nl/en/}}

%
% NB: a more complex sample for affiliations and the mapping to the
% corresponding authors can be found in the file "llncs.dem"
% (search for the string "\mainmatter" where a contribution starts).
% "llncs.dem" accompanies the document class "llncs.cls".
%

%\toctitle{Lecture Notes in Computer Science}
%\tocauthor{Authors' Instructions}
\maketitle

\begin{abstract}
This survey paper gives an overview of spam filtering techniques using either markov 
models or n-grams A short introduction to na\"ive bayesian classifiers is given with 
further explanation about n-grams and markov models and a comparison is given between
these techniques, outlining strengths and weaknesses. Based on the presentation of these
approaches and the performance of models that were presented at TREC conferences 2005-2007
the conclusion can be drawn that N-gram based approaches have surpassed pure markov
models especially for machine learning.
\keywords{markov model, n-gram, spam, filtering, bayes}
\end{abstract}

\section{Introduction}
Spam is a well-known concept in day-to-day e-mail communication, and spam can be 
defined as: ``Unsolicited, unwanted email that was sent indiscriminately, directly 
or indirectly, by a sender having no current relationship with the recipient."\cite{2005overview}

Although this is by no means a new phenomenon, the battle to limit spam is still ongoing. 
The countermeasures taken started out as manual intervention by administrators or a simple 
rule-based filter to block spam from getting delivered. With such efforts being taken, spam
has evolved to evade countermeasures\cite{2006survey}, upon which the countermeasures are evolving 
to defeat this attempt by spam to evade the countermeasures\cite{2007systematic}. It is apparent 
this makes the current spam countermeasures complex and an ever-changing research field.

A spam filter can be defined as an automated technique to identify spam. 
For usual spam filters this results in a likelihood that a certain message is 
spam (or ham [not spam]). A threshold can be used to convert this into a 
binary spam-or-ham decision that takes a balance into account between 
false positives and false negatives. For finding a good balance it is important
to note that classifying several spam messages as ham is excusable; people
can delete them themselves, but classifying a ham message as spam is a much
worse mistake, because it means the recipient will most likely not read the
possibly important message. 

Using statistics or probability theory it is possible to better determine the 
likelihood that a certain message is spam. For this the Na\"ive Bayes Classifier 
is a useful example, allowing for a calculation of spam chances based 
upon multiple sources of evidence (the `Na\"ive' assumption is that 
all the sources are independent, given the class)\cite{2005ngram} 
There are numerous ways to calculate these likelihoods or to pre-process 
a message that's about to be evaluated for spam content. Two of these ways 
are N-gram-based approaches and Markov models, which will both be 
explaned in more detail in subsequent sections.

\section{Related Work}
There are lots of papers that present a survey of techniques of e-mail spam 
filtering like \cite{2007systematic,2009elsevier}. For example, \cite{2007systematic} 
provides, next to a systematic overview of what spam is and how you can 
detect it as such, types of classifiers and ways to filter incoming messages 
among which whitelisting, collaborative filtering and Challenge-Response 
systems. It also discusses automatic learning for fine-tuning the filter, in 
several ways.

In \cite{2009elsevier} several machine learning examples are summarized, 
and each is explained with the results of influential papers on 
the applicable subjects. For instance the work of \cite{medlock2006} 
is presented where bigrams lead to an improved performance for a na\"ive 
bayesian classifier compared to unigrams or, because of performance 
tradeoffs, trigrams.

Other surveys that contain the methods discussed in this paper give an overview 
of a yearly spam-detection contest and the properties of the best-ranking 
approach, based on the filter submissions prior to a TREC conference\cite{2005overview,2006overview,2007overview}. 

\section{Evaluated Methods}
Two of the commonly used approaches to spam filtering will be addressed in this section: Markov models and N-gram-based techniques. For N-grams it is important to first explain the Na\"ive Bayesian Classifier in more detail.

\subsection{Na\"ive Bayes}
Bayes' framework for conditional probabilities has the useful quality that it allows 
for the updating of a probability based on new evidence from multiple sources. In 
this classifier, a given unclassified e-mail message is denoted as $x$ and the 
probability that a given message $x$ belongs to either $s$ (spam) or $h$ (ham) is 
denoted as $P(s|x)$ or $P(h|x)$. Bayes' law of conditional probability now states that 
\begin{equation}
\label{eq:bayes}
P(s|x)=\frac{P(x|s)P(s)}{P(x)}
\end{equation}
where $P(x) = P(x|s)P(s)+P(x|h)P(h)$. Here $P(x)$ is the 
probability that any random message looks like message $x$ and $P(x|s)$ is the 
probability that a message of the spam category looks like $x$.

For computing these last two probabilities the message is split up into tokens. For each
token the chance of occurrance in spam messages and in non-spam messages is recorded 
from a training set of messages. Next it is assumed that occurrances of two seperate 
tokens is statistically independent, and the probability of a spam- or ham-message being
message $x$ can be calculated by multiplying all the chances of the observed token occurrances.

\subsubsection{Token selection}
What kind of tokens are selected is the main difference between different na\"ive 
Bayes classifiers. By default each word can be selected as a token, so that for each
word a seperate chance of occurring in spam and ham messages is calculated. This
means that \texttt{information about the retrieval of lost space objects} and 
\texttt{information retrieval on space stations} would rank the same when compared 
to the subject of \texttt{information retrieval}, indicating that topics of two or 
more words wouldn't be visible in this representation and typical spam subjects 
(\texttt{free money}) would be rated inaccurately because of legitimate uses in sentences 
or messages containing both words in an unrelated way.

\subsection{N-grams}
To reintroduce this interword dependency N adjacent words (N-grams) can be selected as 
a token. In \texttt{information retrieval on space stations} the 2-grams (bigrams) 
\texttt{information retrieval, retrieval on, on space, space stations} would be 
selected. Here you would gain significant topical precision when matching it to 
the document representation of a message that is to be classified as spam or ham, 
but in the process you would lose the ability to classify regular messages that don't
share a topical similarity to a message from the training corpus. This can be solved
again by recording the single words (unigrams) as well as the bigrams. For every additional
set of n-grams that is recorded, a more detailed spam classification can be made.

To include the interaction between nearby words in sentences that don't succeed 
eachother, without including ever longer n-grams, sparse bigrams can be used: a combination 
of two words that occur together in a sentence, seperated by no more than $n$ words, 
without the words occurring inbetween. For example, \texttt{I think therefore I am} 
with $n=2$ would generate \texttt{I think, I therefore, I I, think therefore, think 
I, think am, therefore I, therefore am, I am} as sparse bigrams. The greater $n$, 
the less interaction there can be assumed to be among the nearby words. To make use 
of the distance between the two words in the bigram, the distance can be recorded 
in the bigram by putting a \texttt{?} for every omitted word, e.g. \texttt{think+?+?+am}. 
This method is called Orthogonal sparse bigrams\cite{2007systematic} (OSB).

Finally n-grams can be based on characters as well as words. Character 4-grams for the 
previous example would be \texttt{I\_th, \_thi, thin, hink, ink\_, nk\_t} and so on, where 
\texttt{\_} represents a space.

\subsection{Markov Models}
A Markov model is a system where the next value is predicted based on the preceding $k$ values. 
For text messages, the characters are used as values, so a markov model would be modeling the
probability that after the characters \texttt{to\_thin} the character \texttt{k} will appear
next \cite{2005markovcharacter}. When matching a new message to the model, the probability of
each character of the message is assessed based on the preceding characters. If the $k$ 
preceding characters don't appear in the training set, the longest preceding set of
characters that appears in the model is taken instead for determining the probability of the 
actual character that follows.

The resulting chances are combined as independent probabilities. Because the chance of each 
character occurance on each spot is $0<p<1$, this probability converges to zero for long 
documents, making a normalizing factor necessary, based on the document length.\cite{2005markovcharacter} The 
normalizing factor does not influence the classification outcome, but rather makes document 
probabilities independent of document length, which is useful for threshold-based decisions.

A classification based such a character Markov model can also be adapted to incorporate context 
chains of different levels, discounted by the amount of preceding characters. A match where 
the preceding eight characters are the same should be valued higher than a match where only the 
previous two characters are the same.

The granularity of such markov chains can be made arbitrarily high, from whole sentences 
as tokens to single bits in the bitstream. For example, in \cite{2005markovstatistical} 
a dynamic Markov compression is used as a model where each bit in the message 
bitstream is given a probability based on the preceding bits. 

\subsection{Using N-grams within Markov models}
For building a Markov model, tokens need to be selected for building the prediction chains. So 
far for Markov models, character tokens were assumed but ofcourse word tokens or n-word tokens 
(n-grams) are also possible. Usual Markov models use around 5 context characters because
increasing the context order beyond that usually results in a decrease of performance\cite{teahan2000}.
Because of this, the results for Markov model based techniques can not reliably be evaluated against
N-gram based techniques where Markov models are also using N-grams. For evaluation the approaches
need to be judged on merit, when comparing techniques using Markov models with (all) other techniques,
and when comparing techniques using unigrams to techniques using bi- or N-grams.

\section{Comparison}
Where the previous section gives an introduction to both approaches, this section focuses on the 
benefits and downsides of either approach. The approaches can be judged on memory requirements but 
also on the speed of classification (efficiency) and on the correctness of classifications (effectiveness).

\subsection{Memory}
A clear downside of n-grams is the increased number of features that have to be stored: 
the number of possible combinations of words is exponential with regard to $n$. For 
trigrams it already shows that the additional computations required don't weigh up 
to the additional detail that is captured, because topics consisting of three 
words (like \texttt{world trade centre}) occur very sparsely as bigrams (e.g. 
\texttt{world trade} and \texttt{trade centre}) when not discussing the same 
topic\cite{2007systematic}.

For Markov models memory is also an issue. In \cite{2005markovcharacter} it was found that the model
needed to be pruned after training on roughly 15.000 sample messages due to memory constraints.
When the maximum number of pre-recorded characters is 6 at most, when estimating around 100 possible
characters, including punctuation, in every position, the maximum number of character 6-grams is $10^{12}$.
Because the number of words in the English language has been estimated to be more than $10^6$ as of 2012\cite{2012english},
this means there are roughly as much character 6-grams as possible bigrams in the English language.

\subsection{Spam detection}
For the TREC 2005 corpora the PPM\cite{2005markovcharacter} (Prediction by Partial Matching) approach 
using a Markov model performed clearly the best in all instances\cite{2007systematic}, leading to it becoming a standard to 
test against for TREC in next years. All bag-of-words approaches (including na\"ive bayesian classifiers using n-grams) were
left far behind. The PPM Markov model was used on the TREC 2005 corpora using either 4, 6 or 8 context characters, and it 
was shown that the nature of the corpus had more influence on PPM performance than the number of context characters. While
6 characters performed overall best out of these three options, the model seems robust for the choice of this parameter\cite{2005markovcharacter}.

For TREC 2006 the general filter effectiveness rose and several other approaches came 
close to PPM. OSBF-Lua\cite{2006osbflua}, which is based on orthogonal sparse bigrams, 
was shown to have several approaches that equaled or even bettered PPM\cite{2007systematic}.
However, OSBF-Lua was considered by the authors to perform this well because of their
enhanced training method.

For TREC 2007 the best performing model was ROSVM\cite{2007rosvm} which uses binary
features and character 4-grams. Its success was also caused by its online active 
learning features, updating the model when new messages have been processed.

For learning features, it is suspected that much improvement can still be made for
the PPM approach that appeared in 2005. All approaches that performed well on TREC
2005 were considered fairly weak in terms of machine learning\cite{2007systematic}.
Because the n-gram based approaches were evolving rapidly in TREC 2006 and TREC 2007,
they surpassed PPM in machine learning. This trend was visible for all compression
models\cite{2007overview}. However, the PPM method has still be improved by using
Dynamic Markov Compression and using only a 2500 byte prefix of the message for
classification rather than the entire messace, which led to improved efficiency and, 
more surprising, to more effectiveness\cite{2007systematic}.

\section{Conclusions}
Spam filters are improving rapidly and so is the best approach to filtering. Where 
Markov compression was the best filter in 2005, more recently it has been surpassed 
by newer models. Meanwhile both N-gram based approaches and Markov models have proven
that they can increase filtering effectiveness. 

Both for N-grams and Markov context chains there is a tradeoff between efficiency 
and effectiveness. For Markov context chains it is known that for more than 5 tokens
of context the efficiency degrades too much compared to the effectiveness gain. 
For N-gram approaches the orthogonal sparse bigrams appeared to have most merit 
for effective spam recognition, providing a good tradeoff between efficiency while 
still giving a good effectiveness.

Based on more recent performances of both models, it can be concluded that N-gram
based approaches like orthogonal sparse bigrams have surpassed markov models in terms
of both efficiency and effectiveness, as of 2007.

\begin{thebibliography}{4}

%References discussing Markov Models
\bibitem{2005markovstatistical} Bratko, A., Cormack, G.V., Filipic, B., Lynam, R., Zupan, B.: Spam Filtering Using Statistical Data Compression Models. In \emph{Proceedings of TREC}, 2005.
\bibitem{2005markovcharacter} Bratko, A., Filipic, B.: Spam Filtering using Character-level Markov Models: Experiments for the TREC 2005 Spam Track. In \emph{Proceedings of TREC}, 2005.

%References discussing N-grams
\bibitem{2005ngram} Keselj, V., Milios, E., Tuttle, A., Wang, S., Zhang, R.: Spam Filtering using N-gram-based Techniques. In \emph{Proceedings of TREC}, 2005.
\bibitem{2006osbflua} Assaz, F.: OSBF­Lua,­ A text classification module for Lua, The importance of the training method. In \emph{Proceedings of TREC}, 2006.
\bibitem{2007rosvm} Sculley, D., Wachman, G.M.: Relaxed online SVMs in the TREC Spam 
filtering track. In \emph{Sixteenth Text REtrieval Conference}, 2007.

%More survey-like references
\bibitem{2006survey} The CRM114 Team: Seven Hypothesis about Spam Filtering. In \emph{Proceedings of TREC}, 2006.
\bibitem{2007systematic} Cormack, G.V.: Email Spam Filtering: A Systematic Review. In \emph{Foundations and Trends in Information Retrieval archive, Volume 1 Issue 4, April 2007}.
\bibitem{2009elsevier} Guzella, T.S., Caminhas, W.M.: A review of machine learning approaches to Spam Filtering. Expert Systems with Applications, \emph{Elsevier, Volume 36, Issue 7, September 2009, Pages 10206–10222}


%TREC Spam Track Overview
\bibitem{2005overview} Cormack, G.V., Lynam, T., TREC 2005 Spam Track Overview. In \emph{Proceedings of TREC}, 2005.
\bibitem{2006overview} Cormack, G.V.: TREC 2006 Spam Track Overview. In \emph{Proceedings of TREC}, 2006.
\bibitem{2007overview} Cormack, G.V.: TREC 2007 Spam Track Overview. In \emph{Proceedings of TREC}, 2007.

%Other references
\bibitem{medlock2006} Medlock, B.: An adaptive, semi-structured language model approach to spam filtering on a new corpus. In \emph{Proceedings of the third conference on email and anti-spam.}
\bibitem{2012english} Global Language Monitor: Number of Words in the English Language: 1,010,649.7 \url{http://www.languagemonitor.com/global-english/number-of-words-in-the-english-language-1008879/}
\bibitem{teahan2000} Teahan, W. J.: Text classification and segmentation using minimum cross-entropy. In \emph{Proceedings of RIAO-00, 6th International Conference “Recherche d’Information Assistee par Ordinateur”}, 2000

\end{thebibliography}

\end{document}
